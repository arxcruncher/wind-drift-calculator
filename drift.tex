\documentclass[a4paper,10pt]{article}
\usepackage[utf8]{inputenc}

\usepackage{gensymb}
\usepackage{geometry}
\geometry{margin=3cm}

\usepackage{tikz}
\usetikzlibrary{decorations.markings}
\usepackage{amsmath}
\usepackage{aircraftshapes}
\usepackage{subfig}

%opening
\title{Wind Drift Calculation}
\author{Karel De\,Vogeleer}


\newcommand{\Tt}{T_\mathrm{track}}
\newcommand{\Tc}{T_\mathrm{course}}
\newcommand{\Tw}{T_\mathrm{wind}}
\newcommand{\vGS}{v_\mathrm{GS}}
\newcommand{\vw}{v_\mathrm{wind}}
\newcommand{\vIAS}{v_\mathrm{AIS}}

\begin{document}

\maketitle

\section{Introduction}

Imagine an aircraft flying 95\,kts at 2000\,ft wanting to maintain a true track heading of 20\degree and a wind blows 12\,kts from 120\degree.
What heading does the aircraft have to maintain to compensate for the wind drift?
This document explains how to calculate the heading when subject to a wind drift.

Two different calculation approaches are explained, via \emph{Pythagoras's theorem} and the \emph{cosine rule}.
We will see that the former is a tad simpler than the latter, but both equally accurate.

Let's put forward a few variables to describe our wind drift problem:
\begin{itemize}
 \item $\vec{a}$ is the aircraft velocity,
 \item $\vec{w}$ is the wind velocity,
 \item $\vec{r}$ is the resulting velocity $\vec{a}+\vec{w}$.
\end{itemize}
A velocity\footnote{Wikipedia: the \emph{velocity} of an object is the rate of change of its position with respect to a frame of reference, and is a function of time. Velocity is equivalent to a specification of its speed and direction of motion (e.g. 60\,km/h to the north).} has a length $\|\cdot\|$ and a given direction or heading $\theta(\cdot)$.

In the context of speeds and an aircraft subject to wind, $\|\vec{a}\|$ is also referred to as the \emph{indicated airspeed} (IAS), and $\|\vec{r}\|$ is also known as the \emph{ground speed} (GS), the \emph{true track} ($T_\mathrm{t}$) is the heading of $\vec{r}$ and the \emph{true course} ($T_\mathrm{c}$) is the heading of $\vec{a}$.
The speeds can be in any unit, e.g., mph, kts, m/s or km/h, as long as the same unit is used consistently.

For the example in the introduction, the aircraft's true track is $T_\mathrm{t}=\theta(\vec{r})=20\degree$ and flying a ground speed of $\mathrm{GS}=\|\vec{r}\|=96.35$\,kts.
The wind blows in the direction of  $\theta(\vec{w})=300\degree$ blowing at $\|\vec{w}\|=12$\,kts.
This results in a true course heading $T_\mathrm{t}=\theta(\vec{a})=27.15\degree$ and an indicated airspeed of $\mathrm{IAS}=\|\vec{a}\|=95$\,kts.

\begin{figure}%
    \centering
    \subfloat[velocities and angles]{
    
     \centering
 \begin{tikzpicture}
  %\draw[thin,gray!40] (-2,-2) grid (2,2);
  \draw[gray,->,dashed] (0,-0.5) -- (0,4);
  \draw[gray,->,dashed] (-0.5,0) -- (4,0);
  \draw[gray,->,dashed] (4,3.5) -- (4,5);
  \draw[gray,->,dashed] (3.5,4) -- (5,4);
  \draw (1.9,2) node[above] {$\vec{a}$};
  \draw (3,1) node[below] {$\vec{r}$};
  \draw (5.1,3) node[right] {$\vec{w}$};
\begin{scope}[thick,decoration={markings,mark=at position 0.5 with {\arrow{>}}}] 
  \draw[postaction={decorate}] (0,0) -- (4,4);
\end{scope}
\begin{scope}[thick,decoration={markings,mark=at position 0.55 with {\arrow{>}},mark=at position 0.45 with {\arrow{>}},mark=at position 0.5 with {\arrow{>}}}]
  \draw[postaction={decorate}] (0,0) -- (6,2);
\end{scope}
\begin{scope}[thick,decoration={markings,mark=at position 0.55 with {\arrow{>}},mark=at position 0.45 with {\arrow{>}}}]
  \draw[postaction={decorate}] (4,4) -- (6,2);
\end{scope}
  \node [aircraft top,fill=black,minimum width=0.75cm, rotate=45] at (0,0) {};
  \draw[gray,->] (0,1) arc (90:19:1);
  \draw[gray,->] (0,1.5) arc (90:45.5:1.5);
  \draw[gray,->] (4,4.5) arc (90:-43:0.5);
  \draw[gray] (0.75,1.3) node[above] {$\theta(\vec{a})$};
  \draw[gray] (-0.4,0.75) node[above] {$\theta(\vec{r})$};
  \draw[gray] (4.75,4.25) node[above] {$\theta(\vec{w})$};
\end{tikzpicture}

    
    }%
    \qquad
    \subfloat[speeds and drift]{
    
     \centering
 \begin{tikzpicture}
  \draw[gray,->,dashed] (0,-0.5) -- (0,4);
  \draw[gray,->,dashed] (-0.5,0) -- (4,0);
  \draw[gray,dashed] (4,4) -- (4.75,1.58);
  \draw[gray] (4.4,1.57) -- (4.65,1.65) -- (4.57,1.9);
  \draw (1.8,2) node[above] {$\|\vec{a}\|$};
  \draw[blue] (3.1,1) node[below] {$\|\vec{r}_\mathrm{a}\|$};
  \draw[red] (5.5,1.75) node[below] {$\|\vec{r}_\mathrm{w}\|$};
  \draw (5.1,3) node[right] {$\|\vec{w}\|$}; 
  \draw[->] (0,0) -- (4,4);
  \draw[->,blue] (0,0) -- (4.75,1.58);
  \draw[->,red] (4.75,1.58) -- (6,2);
  \draw[->] (4,4) -- (6,2);
  \node [aircraft top,fill=black,minimum width=0.75cm, rotate=45] at (0,0) {};
  \draw[gray,->] (5.283956,2.682788) arc (137.5:196:1);
  \draw[gray,->] (1.069876,1.051364) arc (45.5:21:1.5);
  \draw[gray] (3.85,2.2) node[above] {$\|\vec{w}_\mathrm{h}\|$};
  \draw[gray] (4.9,2.1) node[above] {$\beta$};
  \draw[gray] (1.55,0.65) node[above] {$\alpha$};
\end{tikzpicture}
    \label{fig:overview:b}
    }%
\caption{Example of the true velocity $\vec{r}$ of an aircraft subject to wind $\vec{w}$ when following a course $\vec{a}$. The \emph{true track} $T_\mathrm{t}$ is $\theta(\vec{r})$, while the \emph{true course} $T_\mathrm{c}$ equals $\theta(\vec{a})$ and the winds blows in the direction $T_\mathrm{w}$. $\|\vec{a}\|$ represents the \emph{indicated airspeed} AIS and the \emph{ground speed} GS is given by $\|\vec{r}\|$, which can be decomposed in $\vec{r}_\mathrm{a} + \vec{r}_\mathrm{w}$.}
\label{fig:overview}
\end{figure}


\section{Calculation}

The aim of the wind drift calculation is to obtain (1) our \emph{ground speed} (GS) $\|\vec{r}\|$, and (2) find our \emph{wind drift angle} $\alpha$.
Figure~\ref{fig:overview:b} shows the variables that we are going to use in the calculations.
Via the wind drift angle $\angle$ we are able to calculate the heading to steer
\begin{equation}
 T_\mathrm{t}=T_\mathrm{c}+\alpha\quad\text{or}\quad\theta(\vec{a}) = \theta(\vec{r}) + \alpha,
\end{equation}
\begin{equation}
 \text{GS}= \|\vec{r}\| = \|\vec{r}_\mathrm{a}\| + \|\vec{r}_\mathrm{w}\|.
\end{equation}

$\beta$ is an angle given by the initial condition of the problem.
We know the heading of the wind $\theta(\vec{w})$ and the direction in which we want to move $T_\mathrm{t}=\theta(\vec{r})$.
In the case of Figure~\ref{fig:overview}, the angle between the wind and the true track $T_\mathrm{t}$ is simply
\begin{equation}
 \beta = \theta(\vec{w}) - \theta(\vec{r}),
\end{equation}
as the left-hand side and the right-hand side of this equation are, in fact, vertically opposite angles. 

Let's start the drift calculate for the example as shown in Figure~\ref{fig:overview}.
Later on we'll generalise the solution for cases where the wind blowes from other quadrants.

\subsection{Pythagoras}

The Pythagoras is fairly simple, it requires twice the application of the sine rule, and once Pythagoras's theorem.
The key to the solution is to calculate $\|\vec{w}_\mathrm{h}\|$ and $\|\vec{r}_\mathrm{w}\|$.
$\vec{w}_\mathrm{h}$ and $\vec{r}_\mathrm{w}$ form the wind velocity 
\begin{equation}
 \vec{w}= \vec{w}_\mathrm{h}+\vec{r}_\mathrm{w},
\end{equation}
where $\vec{r}_\mathrm{w}$ is the projection of the wind velocity on the true track velocity.
Knowing $\|\vec{w}_\mathrm{h}\|$, $\|\vec{a}\|$, $\beta$ and $\|\vec{w}\|$, one can calculate $\alpha$, $\|\vec{r}\|_\mathrm{a}$ and $\|\vec{r}\|_\mathrm{b}$.

$\|\vec{w}_\mathrm{h}\|$ is obtained by applying the sine rule to $\beta$:
\begin{equation}
 \|\vec{w}_\mathrm{h}\| = \|\vec{w}\| \sin(\beta).\label{eq:wh}
\end{equation}
Now $\|\vec{w}_\mathrm{h}\|$ can be used to calculate $\alpha$ as we also know $\|\vec{a}\|$,
\begin{equation}
 \sin (\alpha) = \frac{\|\vec{w}_\mathrm{h}\|}{\|\vec{a}\|}.
\end{equation}
Replacing $\|\vec{w}_\mathrm{h}\|$ from Equation \ref{eq:wh}, and isolating $\alpha$ yields the drift angle $\alpha$:
\begin{equation}
 \alpha = \arcsin\left(\frac{\|\vec{w}\| \sin(\beta)}{\|\vec{a}\|}\right).
\end{equation}

To calculate $\|\vec{r}\|$ we need to know $\|\vec{r}_\mathrm{b}\|$ and $\|\vec{w}_\mathrm{w}\|$.
$\|\vec{r}_\mathrm{b}\|$ can be obtained by applying Pythagoras' theorem:
\begin{equation}
 \|\vec{r}_\mathrm{a}\| = \sqrt{\|\vec{a}\|^2-\|\vec{w}_\mathrm{h}\|^2}\label{eq:ra}
\end{equation}
Also, $\|\vec{r}_\mathrm{w}\|$ is obtained similar to $\|\vec{w}_\mathrm{h}\|$:
\begin{equation}
 \|\vec{r}_\mathrm{w}\| = \|\vec{w}\| \cos(\beta).\label{eq:rw}
\end{equation}
Now that we know both $\|\vec{r}_\mathrm{a}\|$ and $\|\vec{r}_\mathrm{w}\|$ we can calculate $\|\vec{r}\|$
\begin{equation}
 \|\vec{r}\| = \|\vec{r}_\mathrm{a}\| + \|\vec{r}_\mathrm{w}\|.\label{eq:r}
\end{equation}
Merging Equation~\ref{eq:ra} and Equation~\ref{eq:rw} in Equation~\ref{eq:r} yields the groundspeed
\begin{equation}
% \|\vec{r}\| &= \arctan\left(\frac{\|\vec{w}_\mathrm{h}\|}{\|\vec{a}\|}\right) + \|\vec{w}\| \cos(\beta)\\
 \|\vec{r}\| = \sqrt{\|\vec{a}\|^2-\left(\|\vec{w}\| \sin(\beta)\right)^2} + \|\vec{w}\| \cos(\beta).
\end{equation}
Alternatively, the groundspeed can be calculated in function of the the wind drift angle
\begin{equation}
% \|\vec{r}\| &= \arctan\left(\frac{\|\vec{w}_\mathrm{h}\|}{\|\vec{a}\|}\right) + \|\vec{w}\| \cos(\beta)\\
 \|\vec{r}\| = \|\vec{a}\|\cos(\alpha) + \|\vec{w}\| \cos(\beta).
\end{equation}

\subsection{Cosine Rule}

The cosine rule is a bit more complicated as we have to deal with the roots of a quadratic polynomial.
Let's get started by writing the cosine rule for $\beta$:
\begin{equation}
 \|\vec{a}\|^2 = \|\vec{w}\|^2 + \|\vec{r}\|^2 - 2 \|\vec{w}\| \|\vec{r}\| \cos(\beta)
\end{equation}
We are looking to find $\|\vec{r}\|$. In fact, if we look closer, this equation is a quadratic in $\|\vec{r}\|$:
\begin{equation}
 0 =  \|\vec{r}\|^2 - 2 \|\vec{w}\| \cos(\beta) \|\vec{r}\| + (\|\vec{w}\|^2 - \|\vec{a}\|^2).
\end{equation}
We can obtain $\|\vec{r}\|$ by finding the roots of the above equation.
Let's define $c_*$ as the coefficients of the second order polynomial:
\begin{subequations}
\begin{align}
 c_2 &= 1\\
 c_1 &= - 2 \|\vec{w}\| \cos(\beta)\\
 c_0 &= \|\vec{w}\|^2 -\|\vec{a}\|^2
\end{align}
\end{subequations}
The well known solution for a quadratic equation is as follows
\begin{equation}
 \|\vec{r}\| = \frac{-c_1\pm\sqrt{c_1^2-4c_2c_0}}{2c_2}.
\end{equation}
Inserting $c_0$, $c_1$ and $c_2$ yields the groundspeed
\begin{align}
 \|\vec{r}\| = \|\vec{w}\| \cos(\beta)\pm\sqrt{\frac{\|\vec{w}\| \cos(\beta)^2}{2}-(\|\vec{w}\|^2 -\|\vec{a}\|^2)}.
\end{align}
However, we obtain two roots, a negative and a positive one.
Only the positive root applies to our real world.
This corresponds to the summation case, and so we obtain the unambiguous groundspeed
\begin{align}
 \|\vec{r}\| = \|\vec{w}\| \cos(\beta)-\sqrt{\frac{\|\vec{w}\| \cos(\beta)^2}{2}-(\|\vec{w}\|^2 -\|\vec{a}\|^2)}.
\end{align}

At this point we know $\|\vec{a}\|$, $\|\vec{w}\|$, and $\|\vec{r}\|$.
This allows us to calculate the wind drift angle $\alpha$.
Applying the cosine rule once more
\begin{equation}
 \|\vec{w}\|^2 = \|\vec{r}\|^2 + \|\vec{a}\|^2 - 2 \|\vec{r}\|\|\vec{a}\| \cos(\alpha),
\end{equation}
and isolating $\alpha$ give us
\begin{equation}
 \alpha = \arccos\left(\frac{\|\vec{w}\|^2-(\|\vec{r}\|^2 + \|\vec{a}\|^2)}{- 2 \|\vec{r}\|\|\vec{a}\|}\right).
\end{equation}

%Here we assumed that $\beta$ was a positive value, which is necessary for the cosine rule to hold.
%If the 

\section{Example}

TODO

\section{Implementation}

TODO

\end{document}
